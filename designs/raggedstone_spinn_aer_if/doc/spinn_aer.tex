
\def\FullTitle{Interfacing AER devices to SpiNNaker\\ using an FPGA}
\def\ShortTitle{SpiNNaker - AER interface}
\def\Date{08 Mar 2017}
\def\Version{1.2}
\def\Author{Luis A. Plana}
\def\Email{luis.plana@manchester.ac.uk}

\input{appnote.tex}

\section{Introduction}


This note describes the implementation of an interface between
Address Event Representation (AER) devices and the SpiNNaker system,
using an FPGA.


\section{AER interface}

The Address Event Representation (AER) interface is an asynchronous
interface that implements a 4-phase handshake protocol with active-low
request ($\overline{REQ}$) and acknowledge ($\overline{ACK}$)
signals. Events are transmitted in parallel, 16-bit words. The logic
levels 0 and 1 are represented by voltages of 0V and 5V respectively,
and most devices will operate correctly if driven by 3.3V
signals. Additional information can be found in~\cite{AER}.


\section{SpiNNaker link interface}

The SpiNNaker interface is also an asynchronous interface but
implements a 2-phase handshake protocol. This protocol, also known as
Non-Return-to-Zero (NRZ), uses signal transitions, as opposed to
levels, to represent data. Events are transmitted as 40-bit packets
which are sent serially in 4-bit flits starting with the least
significant bits. The logic levels 0 and 1 are represented by voltages
of 0V and 1.8V respectively and will tolerate being driven by 2.5V and
3.3V signals. Most 2.5V devices will operate correctly when driven by
SpiNNaker signals but 3.3V devices will most likely need level
shifters. Additional information can be found in~\cite{spinn-app-7}.


\section{Mapping AER Events to SpiNNaker Multicast Packets}


\image[scale=0.5]{!h}{map3}{map}{Mapping events to SpiNNaker packets.}


As mentioned above, events in SpiNNaker are transmitted as 40-bit
packets. Packets consist of a short 8-bit header, which includes a
parity bit (bit[0]), and a 32-bit routing key. The routing key is used
by the SpiNNaker routers to deliver the packets to the correct
destinations. As is the case for the AER devices, the routing key in
most cases represents the origin of the event and not its destination
or destinations.

In order to map an AER event to a SpiNNaker packet routing key, the
AER device is treated as if it was a SpiNNaker chip generating events
and is assigned a 16-bit \emph{virtual key}. This virtual key must not be
shared with existing chips in the SpiNNaker system. Once the virtual
key is selected, a packet can be constructed as shown in
Figure~\ref{fig:map}. The 16-bit event is mapped according to the
characteristics of the AER device and the way it is treated in the
software.\\


\image[scale=0.5]{!h}{retina_maps}{rmaps}{Silicon retina mappings.}


Figure~\ref{fig:rmaps} shows several mappings used in the example
presented below.  The figure shows, at the top, the event sent by an
AER silicon retina with 128x128 pixels. Four different mappings are
shown below it, corresponding to 4 different sampling options: no
under-sampling (128x128 pixel resolution), four neighbouring pixels
grouped together (64x64 pixel resolution), groups of 16 pixels (32x32
pixel resolution) and, finally, groups of 64 pixels (16x16 pixel
resolution).\\


\image[scale=0.5]{!h}{cochlea_map}{cmaps}{AER-EAR cochlea mapping.}


Figure~\ref{fig:cmaps} shows an AER-EAR cochlea event and how it is
mapped into a SpiNNaker package. The cochlea has 2 ears with 64
channels associated with each ear. Every channel contains 4
neurons. The retina and cochlea mappings have been chosen to simplify
the construction of neuron populations and their placement on
SpiNNaker cores.



\section{Mapping SpiNNaker Multicast Packets to AER Events}


As indicated earlier, SpiNNaker events are transported as 40-bit
SpiNNaker multicast packets. The SpiNNaker packet routing key, which
is used to deliver the packet to its destination, represents the
origin of the event and not its destination. This key usually consists
of two parts: a \emph{virtual key} associated with the SpiNNaker chip
that generated the event and an event identifier, as shown in
Figure~\ref{fig:out_map}. The 16-bit event identifier should be mapped
according to the characteristics of the AER device and the way it is
treated in the software. The most common mapping is a direct map.\\


\image[scale=0.5]{!h}{out_map2}{out_map}{Mapping SpiNNaker packets to events.}



\section{Example Implementation using a Xilinx FPGA}


The mappings shown above are used in a SpiNNaker--AER interface
implemented on a Xilinx Spartan6 FPGA, using a RaggedStone2
board~\cite{ragged}. The board has two banks of I/O pins, one set at
3.3V and the other at 2.5V. The 3.3V bank is used to connect to the
AER devices and the 2.5V bank is used to connect to the SpiNNaker
board.


\image[scale=0.7]{!p}{spinn_2_if_diag}{diag}{Interface Block Diagram.}


Figure~\ref{fig:diag} shows a block diagram of the implementation. The
basic block in the AER-to-SpiNNaker path is the \emph{in\_mapper},
which implements the mappings shown in Figures~\ref{fig:rmaps}
and \ref{fig:cmaps} above. The constructed 40-bit SpiNNaker packet is
delivered to the \emph{dump} module which can dump the packet or pass
it to the \emph{SpiNN driver} module to be serialised and sent through
a SpiNNaker link.

If at any point the SpiNNaker interface \emph{deadlocks}, i.e., stops
accepting packets, the board goes into \emph{dump mode}. In this mode,
indicated by \emph{led4}, the board continues to accept events from
the AER device but dumps them. The SpiNNaker system may need to be
reset and re-booted.

Packets can also be dumped if the SpiNNaker system has not issued
a \emph{start} command to the interface to start sending packets or
has issued a \emph{stop} command to stop packet flow.

The SpiNNaker-to-AER path consists of a \emph{SpiNN receiver} module
that deserialises incoming SpiNNaker packets and passes them to
the \emph{router}, which in turn sends control packets, \emph{i.e.},
commands, to the \emph{control} module and AER packets to
the \emph{out\-mapper} which delivers the mapped event identifier to
the AER device.


\begin{table}[!h]
\begin{center}
\begin{tabular}{| l | r | l |}
\hline
\multicolumn{1}{|c}{command} & \multicolumn{1}{|c|}{32-bit packet key}& \multicolumn{1}{|c|}{function} \\
\hline
start   & 0x$----ffff$ & start packet flow into SpiNNaker \\
stop    & 0x$----fffe$ & stop packet flow into SpiNNaker \\
\hline
\end{tabular}
\caption{Supported interface commands.}
\label{tab:commands}
\end{center}
\end{table}


Interface commands are sent as multicast SpiNNaker packets. The router
uses the packet key to detect commands and pass them to the control
module. All other multicast packets are passed to
the \emph{out\_mapper} module. The router dumps all non-multicast
packets. The supported commands are listed in
table~\ref{tab:commands}.

It is important to note the need to use of synchronisers for the
incoming handshake signals in both interfaces. These asynchronous
signals must be synchronised to the FPGA clock to avoid
metastability. In this case, simple 2-flop synchronisers are used.\\


\begin{table}[!h]
\begin{center}
\begin{tabular}{| r | l |}
\hline
\multicolumn{1}{|c}{device} & \multicolumn{1}{|c|}{function} \\
\hline
sw1           & mode and virtual key selection button \\
sw2           & reset button \\
\hline
led2          & activity indicator (flashes continuously) \\
led3          & reset indicator \\
led4          & dump mode indicator \\
led5          & error indicator \\
\hline
7-seg display & mode and virtual key indicator \\
              & * digits show chosen mode \\
              & * decimal points show chosen virtual key \\
\hline
\end{tabular}
\caption{RaggedStone2 board user interface.}
\label{tab:userif}
\end{center}
\end{table}


A \emph{user interface} module allows the selection of the operating
configuration by the user, who receives information through a
7-segment display and several leds. The functions of the buttons, leds
and displays are shown in Table~\ref{tab:userif}.\\


\begin{table}[!hp]
\begin{center}
\begin{tabular}{| l | l |}
\hline
\multicolumn{1}{|c}{selection} & \multicolumn{1}{|c|}{virtual key} \\
\hline
default   &  0x0200 \\
alternate &  0xfefe \\
\hline
\end{tabular}
\caption{Virtual key.}
\label{tab:addrs}
\end{center}
\end{table}


As mentioned above, the AER device is assigned a virtual key as part
of the mapping process. The board is configured to provide
a \emph{default} virtual key, shown in Table~\ref{tab:addrs}. The user
can select an \emph{alternate} virtual key, also shown in
Table~\ref{tab:addrs}.

The \emph {mode} button (\emph{sw1}) is used to choose an interface
configuration mode. Each configuration mode is a combination of an
input map and a virtual key. The available configuration modes are
shown in Table~\ref{tab:modes}. The user can step through the
different modes by repeatedly pushing the \emph{mode} button. All
input maps discussed above are included and there is also
a \emph{direct} input map which passes the 16-bit AER event directly
to the packet, without any re-structuring. The selected configuration
mode is shown on the 7 segment displays, as shown in
Table~\ref{tab:modes}. Note that the decimal points are used to
indicate the chosen virtual key.


\begin{table}[!ht]
\begin{center}
\begin{tabular}{| l | r | l |}
\hline
\multicolumn{1}{|c}{mode}  & \multicolumn{1}{|c|}{display} & \multicolumn{1}{|c|}{description} \\
\hline
def\_retina\_128 &  128  & default key \& retina with 128x128 pixels \\
def\_retina\_64  &   64  & default key \& retina sub-sampled to 64x64 pixels \\
def\_retina\_32  &   32  & default key \& retina sub-sampled to 32x32 pixels \\
def\_retina\_16  &   16  & default key \& retina sub-sampled to 16x16 pixels \\
\hline
def\_cochlea     & coch  & default key \& cochlea: 2 ears/64 channels/4 neurons\\
\hline
def\_direct      & 0000  & default key \& event passed through without mapping \\
\hline
alt\_retina\_128 & .128  & alternate key \& retina with 128x128 pixels \\
alt\_retina\_64  & . 64  & alternate key \& retina sub-sampled to 64x64 pixels \\
alt\_retina\_32  & . 32  & alternate key \& retina sub-sampled to 32x32 pixels \\
alt\_retina\_16  & . 16  & alternate key \& retina sub-sampled to 16x16 pixels \\
\hline
alt\_cochlea     & c.och & alternate key \& cochlea: 2 ears/64 channels/4 neurons\\
\hline
alt\_direct      & 0.000 & alternate key \& event passed through without mapping \\
\hline
\end{tabular}
\caption{Interface configuration modes.}
\label{tab:modes}
\end{center}
\end{table}




\subsection{Implementation Code}


The interface was implemented using Verilog. The code is part of
\textbf{spI/O}~\cite{spio2014}, a library of FPGA designs and re-usable modules for I/O in SpiNNaker systems. \textbf{spI/O} is available as a
SpiNNakerManchester GitHub repository
(https://github.com/SpiNNakerManchester/spio).



\begin{thebibliography}{10}

\bibitem{AER}
P. H\"{a}fliger, ``CAVIAR Hardware Interface Standards'', Version 2.0, Deliverable D WP7.1b,
\\online: http://www2.imse-cnm.csic.es/caviar/download/ConsortiumStandards.pdf, 2003.

\bibitem{spinn-app-7}
S. Temple, ``AppNote 7 - SpiNNaker Links'', Version 1.0,
\\online: http://solem.cs.man.ac.uk/documentation/spinn-app-7.pdf, 2012.

\bibitem{ragged}
Enterpoint Ltd., ``RaggedStone2 User Manual'', Issue 1.02,
\\online: http://www.enterpoint.co.uk/raggedstone/RaggedStone\_User\_Manual\_Issue\_1\_02.pdf, 2010.

\bibitem{spio2014}
{L.A. Plana}, J.~Heathcote, J.~Pepper, S.~Davidson, J.~Garside, S.~Temple, and
  S.~Furber, ``{spI/O: A library of FPGA designs and reusable modules for I/O
  in SpiNNaker systems.}'' May 2014. [Online]. Available:
  \tt{https://doi.org/10.5281/zenodo.51476}

\end{thebibliography}


\subsection{\itshape Change log:}

\begin{itemize}
\item {\itshape 1.0 - 17apr13 - lap} - initial release - comments to
  {\itshape \Email}
\item {\itshape 1.1 - 03apr14 - lap} - bidirectional interface - comments to
  {\itshape \Email}
\item {\itshape 1.2 - 08mar17 - lap} - router and control modules - comments to
  {\itshape \Email}
\end{itemize}

\end{document}
